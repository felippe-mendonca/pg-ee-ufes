% --------------------------------------------------- %
%	Modelo para elaboração do Projeto de Graduação	  %		
%	do curso de Engenharia Elétrica - UFES			  %
%													  %
%	Adaptado do modelo de trabalho acadêmico 'abntex2'%
%	em 27/10/2016									  %
% --------------------------------------------------- %

\documentclass[
	% -- opções da classe memoir --
	12pt,				% tamanho da fonte
	openright,			% capítulos começam em pág ímpar (insere página vazia caso preciso)
	oneside,			% para impressão em recto e verso. Oposto a oneside
	a4paper,			% tamanho do papel. 
	% -- opções da classe abntex2 --
	chapter=TITLE,		% títulos de capítulos convertidos em letras maiúsculas
	%section=TITLE,		% títulos de seções convertidos em letras maiúsculas
	%subsection=TITLE,	% títulos de subseções convertidos em letras maiúsculas
	%subsubsection=TITLE,% títulos de subsubseções convertidos em letras maiúsculas
	% -- opções do pacote babel --
	english,			% idioma adicional para hifenização
	french,				% idioma adicional para hifenização
	spanish,			% idioma adicional para hifenização
	brazil				% o último idioma é o principal do documento
	]{abntex2}

% --------------------------------------------------- %
% 					Pacotes básicos 				  %
% --------------------------------------------------- %
\usepackage{lmodern}			% Usa a fonte Latin Modern			
\usepackage[T1]{fontenc}		% Selecao de codigos de fonte.
\usepackage[utf8]{inputenc}		% Codificacao do documento (conversão automática dos acentos)
\usepackage{lastpage}			% Usado pela Ficha catalográfica
%\usepackage{indentfirst}		% Indenta o primeiro parágrafo de cada seção.
\usepackage{color}				% Controle das cores
\usepackage{graphicx}			% Inclusão de gráficos
\usepackage{microtype} 			% para melhorias de justificação
\usepackage{amsmath}
\usepackage[table,xcdraw]{xcolor} 
\usepackage{multirow}			% Para usar a tabela gerada no www.tablesgenerator.com
\usepackage{lscape} %serve para inserir página no modo paisagem
\usepackage{pdflscape} %serve para inserir página no modo paisagem
\usepackage{subfig}
\usepackage{subfloat}
% >> Pacotes de citações
\usepackage[brazilian,hyperpageref]{backref}	 % Paginas com as citações na bibl
\usepackage[alf]{abntex2cite}	% Citações padrão ABNT
\usepackage{url}
% >> Configuração dos pacotes de citações
	% ---
	% Configurações do pacote backref
	% Usado sem a opção hyperpageref de backref
	\renewcommand{\backrefpagesname}{Citado na(s) página(s):~}
	% Texto padrão antes do número das páginas
	\renewcommand{\backref}{}
	% Define os textos da citação
	\renewcommand*{\backrefalt}[4]{
		\ifcase #1 %
			Nenhuma citação no texto.%
		\or
			Citado na página #2.%
		\else
			Citado #1 vezes nas páginas #2.%
		\fi}%
 
% >>> Insere a pasta onde estão contidas as figuras <<<
\graphicspath{{imagens/}}

% --------------------------------------------------- %
%		Redefinição e criação de comandos 			  %
% --------------------------------------------------- %
	
% >>> Mudar tamanho da fonte dos capítulos <<<
\renewcommand*{\chapnumfont}{\normalfont\large\bfseries\sffamily}
\renewcommand*{\chaptitlefont}{\normalfont\large\bfseries\sffamily}

\usepackage{titlesec}
\titleformat{\section}
  {\normalfont\normalsize\bfseries}{\thesection}{1em}{}
\titleformat{\subsection}
  {\normalfont\normalsize\bfseries}{\thesubsection}{1em}{}

% >>> Mudar tamanho da fonte das legendas <<<
\usepackage[font=footnotesize]{caption}

% >>> Definição do tipo CRONOGRAMA <<<
% e.g.:
% \begin{cronograma}[!h]
% 		Insira o cronograma aqui! (tabela)
% \eng{cronograma}
\newcommand{\cronogramaname}{Cronograma}
\newcommand{\listofcronogramasname}{Lista de Cronogramas}

\newfloat[chapter]{cronograma}{loc}{\cronogramaname}
\newlistof{listofcronogramas}{loc}{\listofcronogramasname}
\newlistentry{cronograma}{loc}{0}

\counterwithout{cronograma}{chapter}
\renewcommand{\cftcronogramaname}{\cronogramaname\space} 
\renewcommand*{\cftcronogramaaftersnum}{\hfill--\hfill}

% >>> Definição do tipo QUADRO <<<
% e.g.:
% \begin{quadro}[!h]
%  Insira aqui o quadro (tabela)
% \end{quadro} 
\newcommand{\quadroname}{Quadro}
\newcommand{\listofquadrosname}{Lista de Quadros}

\newfloat[chapter]{quadro}{loq}{\quadroname}
\newlistof{listofquadros}{loq}{\listofquadrosname}
\newlistentry{quadro}{loq}{0}

\counterwithout{quadro}{chapter}
\renewcommand{\cftquadroname}{\quadroname\space} 
\renewcommand*{\cftquadroaftersnum}{\hfill--\hfill}

% >>> Comando para inserir a fonte em figuras <<<
% e.g.:
% \begin{figure}[!h]
%	\centering
%	\caption{Legenda da Figura.}
%	\includegraphics[width=0.7\textwidth]{figura.jpg}
%	\source[\citeonline{Referencia}.]
%	\label{fig:label_da_figura}
%  \end{figure}
%
% Obs.: Se utilizar apenas "\source", será inserido
%       "Produção do próprio autor."

\newcommand{\source}[1][Produção do próprio autor.]{\begin{flushleft}\footnotesize Fonte: #1\end{flushleft}}

% --------------------------------------------------- %
%		Informações para personalização da capa		  %
% --------------------------------------------------- %
\newcommand{\universidade}{Universidade Federal do Espírito Santo}
\newcommand{\centro}{Centro Tecnológico}
\newcommand{\departamento}{Departamento de Engenharia Elétrica}
\newcommand{\disciplina}{Projeto de Graduação}
\newcommand{\imprimirINSTITUICAO}{
	\MakeUppercase{\universidade} \\
	\MakeUppercase{\centro} \\
	\MakeUppercase{\departamento} \\
	\MakeUppercase{\disciplina} \\
}
% --------------------------------------------------- %
%		Informações para capa e folha de rosto		  %
% --------------------------------------------------- %
\titulo{Título do Projeto}
\autor{Nome do Aluno}
\local{Vitória-ES}
\data{Dezembro/2016}
\orientador{Prof. Dr. José da Silva}
\coorientador{Profa. Dra. Maria da Penha}
\instituicao{%
	\universidade
  	\par
	\centro
  	\par
	\departamento
	\par
	\disciplina
	\par}
\tipotrabalho{Projeto de Graduação}
% O preambulo deve conter o tipo do trabalho, o objetivo, 
% o nome da instituição e a área de concentração 
\preambulo{Parte manuscrita do Projeto de Graduação do aluno \imprimirautor, apresentado ao Departamento de Engenharia Elétrica do Centro Tecnológico da Universidade Federal do Espírito Santo, como requisito parcial para obtenção do grau de Engenheiro Eletricista.}

% --------------------------------------------------- %
%		Configurações do aspecto final do PDF		  %
% --------------------------------------------------- %
% >> Alterando o aspecto da cor azul
\definecolor{blue}{RGB}{41,5,195}
% Informações do PDF
\makeatletter
\hypersetup{
     	%pagebackref=true,
		pdftitle={\@title}, 
		pdfauthor={\@author},
    	pdfsubject={\imprimirpreambulo},
	    pdfcreator={LaTeX with abnTeX2},
		pdfkeywords={abnt}{latex}{abntex}{abntex2}{trabalho acadêmico}, 
		colorlinks=true,       		% false: boxed links; true: colored links
    	linkcolor=black,          	% color of internal links
    	citecolor=black,        		% color of links to bibliography
    	filecolor=magenta,      		% color of file links
		urlcolor=black,
		bookmarksdepth=4
}
\makeatother

% --------------------------------------------------- %
%		Espaçamentos entre linhas e parágrafos 		  %
% --------------------------------------------------- %
% >> O tamanho do parágrafo é dado por:
\setlength{\parindent}{0cm}
% >> Controle do espaçamento entre um parágrafo e outro:
\setlength{\parskip}{18pt}

% --------------------------------------------------- %
%				Compila o Índice 					  %
% --------------------------------------------------- %
\makeindex

% --------------------------------------------------- %
%				Início do Documento					  %
% --------------------------------------------------- %

\begin{document}

% >>> Seleciona o idioma do documento (conforme pacotes do babel)
% \selectlanguage{english}
\selectlanguage{brazil}

% >>> Retira espaço extra obsoleto entre as frases.
\frenchspacing 

% --------------------------------------------------- %
%				Elementos Pré-Textuais				  %
% --------------------------------------------------- %
% \pretextual

% --------------------------------------------------- %
%						Capa						  %
% --------------------------------------------------- %
% >>> Capa Personalizada
\renewcommand{\imprimircapa}{%
	\begin{capa}%
		\center
		{\ABNTEXchapterfont\bfseries\large\imprimirINSTITUICAO}
			\vspace*{1.5cm}
		\includegraphics*[width=0.25\textwidth]{brasao_ufes.jpg}
			\vspace*{1.5cm} \\
		{\ABNTEXchapterfont\Large\imprimirautor}
				\vspace*{2.5cm} \\
		{\ABNTEXchapterfont\bfseries\Large\imprimirtitulo}
			\vfill
			\vspace*{0.5cm}
		{\large\imprimirlocal}
		\par
		{\large\imprimirdata}
			\vspace*{1cm}
	\end{capa}
}

\imprimircapa

% --------------------------------------------------- %
%					Folha de Rosto 					  %
% --------------------------------------------------- %
% >> O * indica que haverá a ficha bibliográfica

\renewcommand{\imprimirfolhaderosto}{

\begin{folhaderosto}
	\begin{center}
    	{\ABNTEXchapterfont\large\imprimirautor}
		    \vspace*{\fill}\vspace*{\fill}
    	\begin{center}
	    	\ABNTEXchapterfont\bfseries\Large\imprimirtitulo
	    \end{center}
    		\vspace*{\fill}
    		\hspace{.45\textwidth}
	    \begin{minipage}{.5\textwidth}
        	\imprimirpreambulo
	    \end{minipage}
    		\vspace*{\fill}
	    \end{center}  
        \begin{center}
        	% >> Se necessáiro, ajustar os \vspace
        	%\vspace*{0.5cm}
        {\large\imprimirlocal}
        \par
        {\large\imprimirdata}
       		%\vspace*{1cm}
      \end{center}
\end{folhaderosto}
}

\imprimirfolhaderosto

% --------------------------------------------------- %
%					Ficha Catalográfica 			  %
% --------------------------------------------------- %

% Isto é um exemplo de Ficha Catalográfica, ou ``Dados internacionais de
% catalogação-na-publicação''. Você pode utilizar este modelo como referência. 
% Porém, provavelmente a biblioteca da sua universidade lhe fornecerá um PDF
% com a ficha catalográfica definitiva após a defesa do trabalho. Quando estiver
% com o documento, salve-o como PDF no diretório do seu projeto e substitua todo
% o conteúdo de implementação deste arquivo pelo comando abaixo:
%
% \begin{fichacatalografica}
%     \includepdf{fig_ficha_catalografica.pdf}
% \end{fichacatalografica}

%\begin{fichacatalografica}
%	\sffamily
%	\vspace*{\fill}					% Posição vertical
%	\begin{center}					% Minipage Centralizado
%	\fbox{\begin{minipage}[c][8cm]{13.5cm}		% Largura
%	\small
%	\imprimirautor
%	%Sobrenome, Nome do autor
%	
%	\hspace{0.5cm} \imprimirtitulo  / \imprimirautor. --
%	\imprimirlocal, \imprimirdata-
%	
%	\hspace{0.5cm} \pageref{LastPage} p. : il. (algumas color.) ; 30 cm.\\
%	
%	\hspace{0.5cm} \imprimirorientadorRotulo~\imprimirorientador\\
%	
%	\hspace{0.5cm}
%	\parbox[t]{\textwidth}{\imprimirtipotrabalho~--~\imprimirinstituicao,
%	\imprimirdata.}\\
%	
%	\hspace{0.5cm}
%		1. Palavra-chave1.
%		2. Palavra-chave2.
%		2. Palavra-chave3.
%		I. Orientador.
%		II. Universidade xxx.
%		III. Faculdade de xxx.
%		IV. Título 			
%	\end{minipage}}
%	\end{center}
%\end{fichacatalografica}

% --------------------------------------------------- %
%					Folha de Aprovação			      %
% --------------------------------------------------- %
% >>> Após apresentação do trabalho, substitua todo o conteúdo 
% por uma imagem da página assinada pela banca com o comando abaixo:
% \includepdf{folhadeaprovacao_final.pdf}

\begin{folhadeaprovacao}
  \begin{center}
    {\ABNTEXchapterfont\large\imprimirautor}
	\begin{center}
     	\ABNTEXchapterfont\bfseries\Large\imprimirtitulo
    \end{center}
    %\vspace*{\fill}
  \end{center}    
    \imprimirpreambulo
		\vspace{-0.5cm}
    \begin{center}
		\hspace{.45\textwidth}
    \begin{minipage}{.5\textwidth}
        Aprovado em ??, de Dezembro de 2016. \\\\
        \textbf{COMISSÃO EXAMINADORA:}
        \assinatura{\textbf{\imprimirorientador} \\ Instituto Federal do Espírito Santo \\ Orientadora} 
        \assinatura{\textbf{\imprimircoorientador} \\ Universidade Federal do Espírito Santo \\ Coorientadora}
		\assinatura{\textbf{Profa. Dra. Fulana} \\ Universidade Federal do Espírito Santo \\ Examinador}
		\assinatura{\textbf{Prof. Dr. Fulano} \\ Universidade Federal do Espírito Santo \\ Examinador}
    \end{minipage}
	    \vspace*{\fill}
   \end{center}
         
   \begin{center}
      	% >> Se necessáiro, ajustar os \vspace
		%\vspace*{0.5cm}
    	{\large\imprimirlocal}
    	\par
    	{\large\imprimirdata}
   		%\vspace*{1cm}
  \end{center}  
\end{folhadeaprovacao}

% --------------------------------------------------- %
%						Dedicatória				      %
% --------------------------------------------------- %
\begin{dedicatoria}
   \vspace*{\fill}
   \centering
   \noindent
   \textit{ Insira a dedicatória aqui!
   } \vspace*{\fill}
\end{dedicatoria}

% --------------------------------------------------- %
%					Agradecimentos				      %
% --------------------------------------------------- %

\begin{agradecimentos}
	Insira os agradecimentos aqui!
\end{agradecimentos}

% --------------------------------------------------- %
%						Epígrafe			      	  %	
% --------------------------------------------------- %

\begin{epigrafe}
    \vspace*{\fill}
	\begin{flushright}
		Insira a epígrafe aqui!
	\end{flushright}
\end{epigrafe}

% --------------------------------------------------- %
%						Resumo				      	  %	
% --------------------------------------------------- %

% Resumo em português
\setlength{\absparsep}{18pt} % ajusta o espaçamento dos parágrafos do resumo
\begin{resumo}

Insira o resumo aqui!

\textbf{Palavras-chave}: Palavra-chave 1; Palavra-chave 2; ...; Palavra-chave N.
\end{resumo}

%% resumo em inglês
%\begin{resumo}[Abstract]
% \begin{otherlanguage*}{english}
%   \noindent 
%   \textbf{Keywords}: latex. abntex. text editoration.
% \end{otherlanguage*}
%\end{resumo}
 
% --------------------------------------------------- %
%					Lista de Figuras	      	  	  %	
% --------------------------------------------------- %
\renewcommand{\listfigurename}{Lista de Figuras}
\listoffigures*
\cleardoublepage

% --------------------------------------------------- %
%					Lista de Tabelas				  %	
% --------------------------------------------------- %

\pdfbookmark[0]{\listtablename}{lot}
\listoftables*
\cleardoublepage

% --------------------------------------------------- %
%					Lista de Quadros				  %	
% --------------------------------------------------- %

\pdfbookmark[0]{\listofquadrosname}{loq}
\listofquadros*
\cleardoublepage

% --------------------------------------------------- %
%			Lista de Abreviaturas e Siglas			  %	
% --------------------------------------------------- %
\begin{siglas}
  \item[UFES] \textit{Universidade Federal do Espírito Santo}
\end{siglas}

% --------------------------------------------------- %
%					Lista de Símbolos				  %	
% --------------------------------------------------- %

\begin{simbolos}
	\item[]
\end{simbolos}

% --------------------------------------------------- %
%						Sumário						  %	
% --------------------------------------------------- %

\pdfbookmark[0]{\contentsname}{toc}
\tableofcontents*
\cleardoublepage

% --------------------------------------------------- %
%					Elementos Textuais				  %
% --------------------------------------------------- %

\textual

\chapter[Introdução]{Introdução}
% Ajustar esse \vspace de acordo com o necessário
\vspace{-42pt}

Teste de citação: \cite{DepEngEle}

\chapter[Justificativa]{Justificativa}
% Ajustar esse \vspace de acordo com o necessário
\vspace{-42pt}

\chapter[Objetivos]{Objetivos}
% Ajustar esse \vspace de acordo com o necessário
\vspace{-42pt}

\chapter[Metodologia]{Metodologia}
% Ajustar esse \vspace de acordo com o necessário
\vspace{-42pt}

% --------------------------------------------------- %
%				Elementos Pós-Textuais				  %
% --------------------------------------------------- %

\postextual

% --------------------------------------------------- %
%				Referências Bibliográficas		      %
% --------------------------------------------------- %

\bibliographystyle{abnt-alf} % Autor-Data
\renewcommand{\bibname}{Referências Bibliográficas}
\bibliography{bibliografia}

% --------------------------------------------------- %
%						Apêndices				  	  %
% --------------------------------------------------- %

\begin{apendicesenv}
% Imprime uma página indicando o início dos apêndices
\partapendices

% Insira os apêndices aqui em forma de capítulos

\end{apendicesenv}

% --------------------------------------------------- %
%						Anexos						  %
% --------------------------------------------------- %

\begin{anexosenv}
%
%% Imprime uma página indicando o início dos anexos
\partanexos

\end{anexosenv}

% --------------------------------------------------- %
%					Índice Remissivo				  %
% --------------------------------------------------- %
\phantompart
\printindex

\end{document}